\documentclass[10pt]{article}
\usepackage[margin=0.5cm]{geometry}
\usepackage{algorithm2e}
\begin{document}

% Here we demonstrate how algorithms or pseudocode can be typeset using the \verb|algorithm| environment provided by the \verb|algorithm2e| package.

% {\itshape You should not load the \verb|algorithm|, \verb|algpseudocode|, \verb|algcompatible|, \verb|algorithmic| packages if you have already loaded \verb|algorithm2e|.} 

% Note that the command and argument syntax provided by \verb|algorithm2e| are very different from those provided by \verb|algpseudocode|. It is important to know clearly which package that you are using, and then accordingly write the relevant commands with the correct syntax.

% \begin{algorithm}
% $i\gets 10$\;
% \eIf{$i\geq 5$}
% {
%     $i\gets i-1$\;
% }{
%     \If{$i\leq 3$}
%     {
%         $i\gets i+2$\;
%     }
% }
% \end{algorithm}

% Every line in your source code \textbf{must} end with \verb|\;| otherwise your algorithm will continue on the same line of text in the output. Only lines with a macro beginning a block should not end with \verb|\;|.

% The above algorithm example is uncaptioned. If you need a caption for your algorithm, use \verb|\caption{...}| inside the \verb|algorithm| environment.
% You can then use \verb|\label{...}| after the \verb|\caption| so that the algorithm number can be cross-referenced, e.g.~Algorithm~\ref{alg:two} and \ref{alg:three}.

% By default, the \verb|plain| algorithm style is used. But if you prefer lines around the algorithm and caption, you can add the \verb|ruled| package option when loading \verb|algorithm2e|, or write \verb|\RestyleAlgo{ruled}| in your document.

\RestyleAlgo{algoruled}

%% This is needed if you want to add comments in
%% your algorithm with \Comment
\SetKwComment{Comment}{/* }{ */}
\SetAlgoLined

% \begin{algorithm}[hbt!]
% \caption{An algorithm with caption}\label{alg:two}
% \KwData{$n \geq 0$}
% \KwResult{$y = x^n$}
% $y \gets 1$\;
% $X \gets x$\;
% $N \gets n$\;
% \While{$N \neq 0$}{
%   \eIf{$N$ is even}{
%     $X \gets X \times X$\;
%     $N \gets \frac{N}{2} $ \Comment*[r]{This is a comment}
%   }{\If{$N$ is odd}{
%       $y \gets y \times X$\;
%       $N \gets N - 1$\;
%     }
%   }
% }
% \end{algorithm}

% The \verb|algorithm| environment is a \emph{float}, like \verb|table| and \verb|figure|, so you can add float placement modifiers \verb|[hbt!]| after \verb|\begin{algorithm}| if necessary.

% \begin{algorithm}
% \caption{Another algorithm with caption}\label{alg:three}
% \KwData{Write here the required data}
% \KwResult{Write here the expected result}
%  initialization\;
%  \While{While condition}{
%   instructions\;
%   \eIf{condition}{
%    instructions1\;
%    instructions2\;
%    }{
%    instructions3\;
%   }
%  }
% \end{algorithm}

% The \verb|algorithm2e| package also provides a \verb|\listofalgorithms| command that works like \verb|\listoffigures|, but for captioned algorithms:


\begin{algorithm}
\caption{Tirer un échantillon de donneurs}\label{alg:cap}
\KwData{
$geo\_level \neq \emptyset$ \Comment{Les niv géo emboîtés (par exemple: commune, dept, region)}
$l\_similar \neq \emptyset$ \Comment{Une liste décrivant les similarités par ordre de priorité}
$risks \neq \emptyset$ \Comment{ind. à risque décrits par leur identifiant, leur similarité, leur géographie et le niveau géographique le plus fin auquel ils sont considérés comme à risque ($scope\_risk$)}
$donors \neq \emptyset$ \Comment{donneurs potentiels décrits par leur ident, leur similarité et leur géo.}
$risks \subset donors$
}

Initialisation du résultat: $res \gets $ liste vide de longueur $\leq$ nb de $geo\_level$ différents\;
$n\_sim \gets length(l\_similar)$\;
$j \gets 1$\;

\While{$j \leq geo\_level$ et $risks \neq \emptyset$ et $donors \neq \emptyset$}{

$res[[j]] \gets$ liste vide de longueur $\leq$ longueur de la liste $l\_similar$\;

$geo \gets geo\_level[j]$\;

\eIf{$j < length(geo\_level)$}{
   $geo\_sup \gets geo\_level[j+1]$\;
   }{
   $geo\_sup \gets NULL$\;
  }
$risks\_geo \gets risks[, scope\_risk == geo]$\;

$s \gets 1$\;

\While{$nrow(risks\_geo) > 0$ et $s \leq n\_sim$}{

$similar \gets l\_similar[s]$\;
\Comment{Calculer le nb ind. à risq pour chaque croisement des var. de similarité et par entité géo}
$stats\_risks \gets summary\_risks(donors, risks\_geo, similar, geo)$\;

$ns \gets nrow(stats\_risks)$\;
$res[[j]][[similar]] \gets $ Liste vide de longueur $\leq ns$\;
$i \gets 1$\;

\While{$nrow(risks\_geo) > 0$ et $i \leq ns$}{

$stats \gets stats\_risks[i,]$: Grp de similarités * geo traité \;
$g \gets stats[1,geo]$: Entité géo. concernée\;
% $n\_draw \gets stats[1,n\_draw]$: Nb. de donneurs à tirer à l'itér. $i$\;

$concern\_risks \gets merge(risks\_geo, stats, by = similar+geo)$\;
% : ind. à risq partageant similarités et géo traités à l'itér $i$\
$concern\_donors \gets merge(risks\_geo, stats, by = similar)[geo \neq g,]$\;
% : donneurs partageant similarités traités et localisés dans les autres entités géographiques de niveau $geo\_level = geo$ identique et partageant également le même niveau $geo\_sup$

$nr$: nb d'ind. à risq à traiter\;
$nd$: nb de donneurs disponibles\;
$n\_draw$: Nombre de donneurs à tirer\;

\eIf{$nr == 0$}{
    $res[[j]][[similar]][[i]] \gets NULL$\;
}{
\eIf{$nd == 0$}{
    $res[[j]][[similar]][[i]] \gets NULL$\;
}{
$n\_draw \gets min(nd, nr)$\;
$orig \gets sample(concern\_risks[, ident], n\_draw)$\;
% \Comment{Dans la majorité des cas $n\_draw = nr$, mais le sample permet un réordonnancement}
$dest \gets  sample(concern\_donors[, ident], n\_draw)$\;
$res[[j]][[similar]][[i]] \gets data(orig = orig, dest =$\;
}
}

Mettre à jour $donors$ et $risks\_geo$ en retirant les donneurs et les ind. à risque tirés\;
$i \gets i + 1$\;

}

$s \gets s + 1$\;   

}

Mettre à jour $risks$ en retirant les ind. retirés de $risks\_geo$\;
$j \gets j + 1$\;


}



\end{algorithm}




% \listofalgorithms


\end{document}
